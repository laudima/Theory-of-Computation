\section{Cadenas y alfabetos}

\begin{Def}
Un \textit{alfabeto} es un conjunto finito no vació. Para referirnos al alfabeto utilizaremos $\Sigma$ 
\end{Def}

\begin{Def}
Los \textit{símbolos} o \textit{letras} son elementos de $\Sigma$, es decir, de nuestro alfabeto. 
\end{Def}

\begin{Def}
    Una \textit{cadena} (string) es una secuencia finita de elementos de $\Sigma$. Por ejemplo, si $\Sigma = \{0,1\}$, entonces $0011$ es una cadena sobre $\Sigma$.
\end{Def}

\begin{Def}
El \textit{tamaño} de una cadena es la cantidad de símbolos que la componen. Por ejemplo, el tamaño de la cadena $0011$ es $4$. Y lo denotamos como $|0011| = 4$.
\end{Def}

\begin{Def}
A la \textit{cadena vacía} la llamamos $\epsilon$  o $\lambda$, de manera de que $|\lambda| = 0$
\end{Def}

\begin{Def}
La \textit{reversa} de una cadena $w$ se denota como $w^R$ y se obtiene invirtiendo el orden de los símbolos en $w$. Por ejemplo, si $w = 0011$, entonces $w^R = 1100$.
\end{Def}

\begin{Def}
Decimos que $x$ es \textit{prefijo} de $y$,  si existe una cadena $z$ tal que $y = xz$. Por ejemplo, $00$ es prefijo de $0011$, ya que existe $z = 11$ tal que $0011 = 00 \cdot 11$.
\end{Def}
